\chapter{Security}\label{chap:security}

The security of \gravity{} relies on the collision resistance of $F$, $H$, $H^*$, on the undetectability and one-wayness of $F$, and on the pseudo-randomness of $G$.
Security reductions in~\cite[Ch.6]{masters} give lower bounds on the complexity of attacks.
We now describe some concrete attack strategies and \gravity's expected strength against them:

\begin{itemize}

\item \textbf{Find two messages that collide for $H^*$}, because their signatures would be identical.
A generic birthday attack has a complexity of 128 bits for $n=256$.

\item \textbf{Break the non-adaptive subset-resilience of $G$.}
Here again, our choices of parameters guarantee a complexity of at least 128 bits for known generic attacks, see~\cite{subsetres} or~\cite[Ch.4]{masters}. 

\item \textbf{Exploit a collision in $\WOTS$ or $\PORST$ instances}: if two secret values (in any of the $\WOTS$ and $\PORST$ instances) are identical, knowing one allows to forge another.
With $n=256$, this gives 128 bits of security if the secret values are chosen independently and uniformly.
However, our construction with $\AES$ guarantees that all secret values are distinct throughout the construction, because $G$ is in fact a permutation.

\end{itemize}
These security estimates hold both against classical and quantum attacks.

This security level corresponds to NIST's category 2 (find a collision for a 256-bit hash).




