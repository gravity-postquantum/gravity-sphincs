\chapter{Specification}\label{chap:specification}

This chapter includes a formal specification of \gravity. 
For more context and an introduction to hash-based signature schemes, we refer to~\cite{masters,ctrsapaper}.

\section{Parameters}

An instance of \gravity{} requires the following parameters:
\begin{itemize}
\item \textbf{Hash output bit length} $n$, a positive integer
\item \textbf{Winternitz depth} $w$, a power of two such that $w \ge 2$ and $\log_2{w}$ divides $n$
\item \textbf{PORS set size} $t$, a positive power of two
\item \textbf{PORS subset size} $k$, a positive integer such that $k \le t$
\item \textbf{Internal Merkle tree height} $h$, a positive integer
\item \textbf{Number of layers of internal Merkle trees} $d$, a non-negative integer
\item \textbf{Cache height} $c$, a non-negative integer
\item \textbf{Batching height} $b$, a non-negative integer
\item \textbf{Message space} $\calM$, usually a subset of bit strings ${\{0,1\}}^*$
\end{itemize}
%
From these parameters are derived the following values, where $\Bn = {\{0,1\}}^n$ denotes the set of $n$-bit strings:
\begin{itemize}
\item \textbf{Winternitz width} $\ell = \mu + \lfloor \log_2{(\mu(w-1))} / \log_2{w} \rfloor + 1$ where $\mu = n / \log_2{w}$
\item \textbf{PORS set} $T = \{0, \ldots, t-1\}$
\item \textbf{Address space} $\calA = \{0, \ldots, d\} \times \{0, \ldots, 2^{c+dh}-1\} \times \{0, \ldots, \max(\ell, t)-1\}$
\item \textbf{Public key space} $\calPK = \Bn$
\item \textbf{Secret key space} $\calSK = \Bn^2$
\item \textbf{Signature space} $\calSG = \Bn \times \Bn^k \times \Bn^{\le k(\log_2{t} - \lfloor \log_2{k} \rfloor)} \times {(\Bn^\ell \times \Bn^h)}^d \times \Bn^c$
\item \textbf{Batched signature space} $\calSG_B = \Bn^b \times \{0, \ldots, 2^b-1\} \times \calSG$
\item \textbf{Public key size}, of $n$ bits
\item \textbf{Secret key size}, of $2n$ bits
\item \textbf{Maximal signature size}, of\[\sigsz = (1 + k + k(\log_2{t} - \lfloor \log_2{k} \rfloor) + d(\ell+h) + c)n \text{ bits}\]
\item \textbf{Maximal batched signature size}, of $\sigsz + bn + b$ bits
\end{itemize}
%


\section{Primitives}

An instance of \gravity{} needs four primitives, which depend on the parameters $n$ and $\calM$:
\begin{itemize}
\item A length-preserving hash function $F : \Bn \to \Bn$
\item A length-halving hash function $H : \Bn^2 \to \Bn$
\item A pseudorandom function $G : \Bn \times \calA \to \Bn$ (takes as input a seed and an address)
\item A general-purpose hash function $H^* : \calM \to \Bn$
\end{itemize}


\section{Internal Algorithms}

We first define algorithms that are building blocks of \gravity{}.

\subsection{Operations on Addresses}

Within the hyper-tree, each $\WOTS$ and $\PORST$ instance is assigned a unique address in order to generate its secret values on demand.
Each address contains:
\begin{itemize}
\item A layer index $0 \le i \le d$ in the hyper-tree, where $0$ is the root layer, $d-1$ is the last $\WOTS$ layer and $d$ is the $\PORST$ layer;
\item An instance index $j$ in the layer, with $0 \le j < 2^{c+(i+1)h}$ if $i < d$ and $0 \le j < 2^{c+dh}$ if $i = d$;
\item A counter $\lambda$ within the instance, with $0 \le \lambda < \ell$ if $i < d$ and $0 \le \lambda < t$ if $i = d$.
\end{itemize}
%
We define the following functions to manipulate addresses.
\begin{itemize}
\item $\mkaddr : \{0, \ldots, d\} \times \mathbb{N} \to \calA$, which takes as input a layer $i \in \{0, \ldots, d\}$ and an index $j \in \mathbb{N}$ and returns the address $a = (i, j \mod 2^{c+dh}, 0) \in \calA$.
\item $\incaddr : \calA \times \mathbb{N} \to \calA$, which takes as input an address $a = (i, j, \lambda)$ and an integer $x$ and returns the address $a' = (i, j, \lambda+x) \in \calA$ with the counter incremented by $x$.
\end{itemize}

\subsection{L-Tree}

The function $\ltree : \Bn^+ \to \Bn$ takes as input a sequence of hashes $x_i \in \Bn$ and returns the associated L-tree root $r \in \Bn$, defined by recurrence as follows.
\[\begin{cases}
\ltree(x_1) = x_1 \\
\ltree(x_1, \ldots, x_{2i+2}) = \ltree(H(x_1, x_2), \ldots, H(x_{2i+1}, x_{2i+2})) \\
\ltree(x_1, \ldots, x_{2i+3}) = \ltree(H(x_1, x_2), \ldots, H(x_{2i+1}, x_{2i+2}), x_{2i+3})
\end{cases}
\]

\subsection{Winternitz Checksum}

The function $\chksum : \Bn \to {\{0, \ldots, w-1\}}^\ell$ takes as input a hash $x \in \Bn$ and returns $\ell$ integers $x_i$, computed as follows.

\begin{itemize}
\item For $i \in \{1, \ldots, \mu\}$ compute $z_i \gets \substr(x, (i-1)\log_2{w}, \log_2{w})$, where $\substr(x, j, m)$ denotes the substring of $x$ of length $m$ bits starting at bit index $0 \le j < |x|$.
\item For $i \in \{1, \ldots, \mu\}$ interpret $z_i$ as the big-endian encoding of a number $0 \le x_i < w$.
\item Compute the checksum $C = \sum_{i=1}^\mu w-1-x_i$.
\item For $i \in \{\mu+1, \ldots, \ell\}$ compute $x_i = \lfloor C / w^{i-\mu-1} \rfloor \mod w$.
In other words, $(x_{\mu+1}, \ldots, x_\ell)$ is the base-$w$ little-endian encoding of the checksum $C$.
\end{itemize}

\subsection{Winternitz Public Key Generation}

The function $\WOTSgen : \Bn \times \calA \to \Bn$ takes as input a secret $\seed \in \Bn$ and a base address $a \in \calA$, and outputs the associated Winternitz public key $p \in \Bn$, computed as follows.

\begin{itemize}
\item For $i \in \{1, \ldots, \ell\}$ compute the secret value $s_i \gets G(\seed, \incaddr(a, i-1))$.
\item For $i \in \{1, \ldots, \ell\}$ compute the public value $p_i \gets F^{w-1}(s_i)$ where the $F^{w-1}$ denotes the function $F$ iterated $w-1$ times.
\item Compute $p \gets \ltree(p_1, \ldots, p_\ell)$.
\end{itemize}

\subsection{Winternitz Signature}

The function $\WOTSsign : \Bn \times \calA \times \Bn \to \Bn^\ell$ takes as input a secret $\seed \in \Bn$, a base address $a \in \calA$ and a hash $x \in \Bn$, and outputs the associated Winternitz signature $\sigma \in \Bn^\ell$, computed as follows.

\begin{itemize}
\item For $i \in \{1, \ldots, \ell\}$ compute the secret value $s_i \gets G(\seed, \incaddr(a, i-1))$.
\item Compute $(x_1, \ldots, x_\ell) \gets \chksum(x)$.
\item For $i \in \{1, \ldots, \ell\}$ compute the signature value $\sigma_i \gets F^{x_i}(s_i)$.
\end{itemize}

\subsection{Winternitz Public Key Extraction}

The function $\WOTSextract : \Bn \times \Bn^\ell \to \Bn$ takes as input a hash $x \in \Bn$ and a signature $\sigma \in \Bn^\ell$, and outputs the associated Winternitz public key $p \in \Bn$, computed as follows.

\begin{itemize}
\item Compute $(x_1, \ldots, x_\ell) \gets \chksum(x)$.
\item For $i \in \{1, \ldots, \ell\}$ compute the public value $p_i \gets F^{w-1-x_i}(\sigma_i)$.
\item Compute $p \gets \ltree(p_1, \ldots, p_\ell)$.
\end{itemize}

\subsection{Merkle Tree Root}

The function $\mroot_h : \Bn^{2^h} \to \Bn$ takes as input $2^h$ leaf hashes $x_i$, and outputs the associated Merkle tree root $r \in \Bn$.
It is defined by recurrence on $h$ as:

\begin{itemize}
\item $\mroot_0(x_0) = x_0$,
\item $\mroot_{h+1}(x_0, x_1, \ldots, x_{2i}, x_{2i+1}) = \mroot_h(H(x_0, x_1), \ldots, H(x_{2i}, x_{2i+1}))$.
\end{itemize}

\subsection{Merkle Tree Authentication}

The function $\mauth_h : \Bn^{2^h} \times \{0, \ldots, 2^h-1\} \to \Bn^h$ takes as input $2^h$ leaf hashes $x_i$ and a leaf index $0 \le j < 2^h$, and outputs the associated Merkle tree authentication path $(a_1, \ldots, a_h) \in \Bn^h$.
It is defined by recurrence on $h$ as:

\begin{itemize}
\item $\mauth_1(x_0, x_1, j) = a_1 \gets x_{j \oplus 1}$ where $\oplus$ denotes the bitwise XOR operation on non-negative integers,
\item $\mauth_{h+1}(x_0, x_1, \ldots, x_{2i}, x_{2i+1}, j)$ is \[\begin{cases}
        a_1 \gets x_{j \oplus 1} \\
        a_2, \ldots, a_{h+1} \gets \mauth_h(H(x_0, x_1), \ldots, H(x_{2i}, x_{2i+1}), \lfloor j/2 \rfloor)
    \end{cases}\]
\end{itemize}

\subsection{Merkle Tree Root Extraction}

The function $\mextract_h : \Bn \times \{0, \ldots, 2^h-1\} \times \Bn^h \to \Bn$ takes as input a leaf hash $x \in \Bn$, a leaf index $0 \le j < 2^h$ and an authentication path $(a_1, \ldots, a_h) \in \Bn^h$, and outputs the associated Merkle tree root $r \in \Bn$.
It is defined by recurrence on $h$ as:

\begin{itemize}
\item $\mextract_0(x, j) = x$,
\item $\mextract_{h+1}(x, j, a_1, \ldots, a_{h+1}) = \mextract_h(x', \lfloor j/2 \rfloor, a_2, \ldots, a_{h+1})$ where \[x' = \begin{cases}H(x, a_1) & \text{if } j \mod 2 = 0 \\ H(a_1, x) & \text{if } j \mod 2 = 1\end{cases}\]
\end{itemize}

\subsection{Octopus Authentication}

The function $\octauth_h : \Bn^{2^h} \times {\{0, \ldots, 2^h-1\}}^k \to \Bn^* \times \Bn$ takes as input $2^h$ leaf hashes $x_i$ and $1 \le k \le 2^h$ distinct leaf indices $0 \le j_i < 2^h$ sorted in increasing order, and outputs the associated octopus authentication nodes $oct \in \Bn^*$ and the octopus root $r \in \Bn$.
It is defined by recurrence on $h$ as:

\begin{itemize}
\item $\octauth_0(x_0, j_1) = (\emptyset, x_0)$,
\item $\octauth_{h+1}(x_0, x_1, \ldots, x_{2i}, x_{2i+1}, j_1, \ldots, j_k)$ is computed as
%
\[\begin{cases}
j_1', \ldots, j_\kappa' \gets \unique(\lfloor j_1/2 \rfloor, \ldots, \lfloor j_k/2 \rfloor) \\
oct', r \gets \octauth_h(H(x_0, x_1), \ldots, H(x_{2i}, x_{2i+1}), j_1', \ldots, j_\kappa') \\
z_1, \ldots, z_{2\kappa - k} \gets (j_1 \oplus 1, \ldots, j_k \oplus 1) \setminus (j_1, \ldots, j_k) \\
a_1, \ldots, a_{2\kappa - k} \gets (x_{z_1}, \ldots, x_{z_{2\kappa - k}}) \\
oct \gets (a_1, \ldots, a_{2\kappa - k}, oct')
\end{cases}\]
where $\unique()$ removes duplicates in a sequence, and $A \setminus B$ denotes the set difference.
\end{itemize}

In other words, $\octauth_h$ is a collective Merkle tree authentication for multiple leaves, that takes care to remove redundant authentication nodes.

\subsection{Octopus Root Extraction}

The function $\octextract_{h,k} : \Bn^k \times {\{0, \ldots, 2^h-1\}}^k \times \Bn^* \to \Bn \cup \{\bot\}$ (with $1 \le k \le 2^h$) takes as input $k$ leaf hashes $x_i \in \Bn$, $k$ leaf indices $0 \le j_i < 2^h$ and an authentication octopus $oct \in \Bn^*$, and outputs the associated Merkle tree root $r \in \Bn$, or $\bot$ if the number of hashes in the authentication octopus is invalid.
It is defined by recurrence on $h$ as:

\begin{itemize}
\item $\octextract_{0,1}(x_1, j_1, oct) = \begin{cases}x_1 & \text{if } oct = \emptyset \\ \bot & \text{otherwise} \end{cases}$,
\item $\octextract_{h+1,k}(x_1, \ldots, x_k, j_1, \ldots, j_k, oct)$ is computed as
%
\[\begin{cases}
j_1', \ldots, j_\kappa' \gets \unique(\lfloor j_1/2 \rfloor, \ldots, \lfloor j_k/2 \rfloor) \\
L \gets \octlayer((x_1, j_1), \ldots, (x_k, j_k), oct) \\
\bot & \text{if } L = \bot \\
\octextract_{h,\kappa}(x'_1, \ldots, x'_\kappa, j_1', \ldots, j_\kappa', oct') & \text{if } L = (x'_1, \ldots, x'_\kappa, oct')
\end{cases}\]
\end{itemize}
%
where $\octlayer()$ is defined by recurrence as:
%
\begin{itemize}
\item $\octlayer(x_1, j_1, oct) = \begin{cases}
        \bot & \text{if } oct = \emptyset \\
        H(x_1, a), oct' & \text{if } oct = (a, oct') \wedge j_1 \mod 2 = 0 \\
        H(a, x_1), oct' & \text{if } oct = (a, oct') \wedge j_1 \mod 2 = 1
    \end{cases}$
\item $\octlayer(x_1, j_1, x_2, j_2, \ldots, x_k, j_k, oct)$ is
%
\[\begin{cases}
    H(x_1, x_2), \octlayer(x_3, j_3, \ldots, x_k, j_k, oct) & \text{if } j_1 \oplus 1 = j_2 \\
    \bot & \text{if } j_1 \oplus 1 \ne j_2 \wedge oct = \emptyset \\
    H(x_1, a), \octlayer(x_2, j_2, \ldots, x_k, j_k, oct') & \text{if } oct = (a, oct') \wedge j_1 \mod 2 = 0 \\
    H(a, x_1), \octlayer(x_2, j_2, \ldots, x_k, j_k, oct') & \text{if } oct = (a, oct') \wedge j_1 \mod 2 = 1
\end{cases}\]
\end{itemize}

In other words, $\octlayer()$ consumes authentication values from the octopus $oct$ along with nodes $x_i$ and their indices $j_i$ to obtain nodes at the upper layer.

\subsection{PRNG to Obtain a Random Subset}

The function $\PORSsf : \Bn \times \Bn \to \mathbb{N} \times T^k$ takes as input a salt $s \in \Bn$ and a hash $x \in \Bn$, and outputs a hyper-tree index $\lambda \in \mathbb{N}$ and $k$ distinct indices $x_i$, computed as follows.

\begin{itemize}
\item Compute $g \gets H(s, x)$.
\item Let $a \gets \mkaddr(0,0)$.
\item Compute $b \gets G(g, a)$ and interpret it as the big-endian encoding of an integer $\beta \in \{0, \ldots, 2^n-1\}$.
\item Compute $\lambda \gets \beta \mod 2^{c+dh}$. In other words, $\lambda$ is the big-endian interpretation of the $c+dh$ last bits of the block $b$.
\item Initialize $X \gets \emptyset$ and $j \gets 0$.
\item While $|X| < k$ do the following:
    \begin{itemize}
    \item increment $j \gets j+1$,
    \item compute $b \gets G(g, \incaddr(a,j))$,
    \item split $b$ into $\nu = \lfloor n / 32 \rfloor$ blocks of $32$ bits, as $b_i = \substr(b, 32(i-1), 32)$,
    \item for $i \in \{1, \ldots, \nu\}$ interpret $b_i$ as the big-endian encoding of an integer $\overline{b_i} \in T$ (as $\overline{b_i} = b_i \mod t$),
    \item for $i \in \{1, \ldots, \nu\}$, if $|X| < k$ compute $X \gets \unique(X, \overline{b_i})$.
    \end{itemize}
\item Compute $(x_1, \ldots, x_k) \gets \sorted(X)$.
\end{itemize}

\subsection{PORST Signature}

The function $\PORSTsign : \Bn \times \calA \times T^k \to \Bn^k \times \Bn^* \times \Bn$ takes as input a secret $\seed \in \Bn$, a base address $a \in \calA$ and $k$ sorted indices $x_i \in T$, and outputs the associated PORST signature $(\sigma, oct) \in \Bn^k \times \Bn^*$ and PORST public key $p \in \Bn$, computed as follows.

\begin{itemize}
\item For $i \in \{1, \ldots, t\}$ compute the secret value $s_i \gets G(\seed, \incaddr(a, i-1))$.
\item For $j \in \{1, \ldots, k\}$ set the signature value $\sigma_j = s_{x_j}$.
\item Compute the authentication octopus and root as\[oct, p \gets \octauth_{\log_2{t}}(s_1, \ldots, s_t, x_1, \ldots, x_k)\]
\end{itemize}

\subsection{PORST Public Key Extraction}

The function $\PORSTextract : T^k \times \Bn^k \times \Bn^* \to \Bn \cup \{\bot\}$ takes as input $k$ indices $x_i \in T$ and a PORST signature $(\sigma, oct) \in \Bn^k \times \Bn^*$, and outputs the associated PORST public key $p \in \Bn$, or $\bot$ if the authentication octopus is invalid, computed as follows.

\begin{itemize}
\item Compute the octopus root $p \gets \octextract_{\log_2{t},k}(\sigma, x_1, \ldots, x_k, oct)$.
\end{itemize}


\section{Signature Scheme}

We now specify the $(\calKG, \calS, \calV)$ algorithms for \gravity{}, as well as batched variants $(\calKG, \calS_B, \calV_B)$.
To simplify, we specify them without secret key caching by the signer.
Indeed, this caching optimization is internal to the signer~--~to increase signing speed~--~and does not change the public results (public key, signature).
We discuss this optimization in \S\ref{sec:caching}.

\subsection{Key Generation}

$\calKG$ takes as input $2n$ bits of randomness and outputs the secret key $sk \in \Bn^2$ and the public key $pk \in \Bn$.

\begin{itemize}
\item Generate the secret key from $2n$ bits of randomness $sk = (\seed, \salt) \getsr \Bn^2$.
\item For $0 \le i < 2^{c+h}$ generate a Winternitz public key
    \[x_i \gets \WOTSgen(\seed, \mkaddr(0, i))\]
\item Generate the public key $pk \gets \mroot_{c+h}(x_0, \ldots, x_{2^{c+h}-1})$.
\end{itemize}

\subsection{Signature}

$\calS$ takes as input a hash $m \in \Bn$ and a secret key $sk = (\seed, \salt)$, and outputs a signature computed as follows.

\begin{itemize}
\item Compute the public salt $s \gets H(\salt, m)$.
\item Compute the hyper-tree index and random subset as $j, (x_1, \ldots, x_k) \gets \PORSsf(s, m)$.
\item Compute the $\PORST$ signature and public key
    \[(\sigma_d, oct, p) \gets \PORSTsign(\seed, \mkaddr(d, j), x_1, \ldots, x_k)\]
\item For $i \in \{d-1, \ldots, 0\}$ do the following:
    \begin{itemize}
    \item compute the $\WOTS$ signature $\sigma_i \gets \WOTSsign(\seed, \mkaddr(i, j), p)$,
    \item compute $p \gets \WOTSextract(p, \sigma_i)$,
    \item set $j' \gets \lfloor j / 2^h \rfloor$,
    \item for $u \in \{0, \ldots, 2^h-1\}$ compute the $\WOTS$ public key
        \[p_u \gets \WOTSgen(\seed, \mkaddr(i, 2^h j' + u))\]
    \item compute the Merkle authentication $A_i \gets \mauth_h(p_0, \ldots, p_{2^h-1}, j - 2^h j')$,
    \item set $j \gets j'$.
    \end{itemize}
\item For $0 \le u < 2^{c+h}$ compute the $\WOTS$ public key
    \[p_u \gets \WOTSgen(\seed, \mkaddr(0, u))\]
\item Compute the Merkle authentication
    \[(a_1, \ldots, a_{h+c}) \gets \mauth_{h+c}(p_0, \ldots, p_{2^{h+c}-1}, 2^h j)\]
\item Set $A_c \gets (a_{h+1}, \ldots, a_{h+c})$.
\item The signature is $(s, \sigma_d, oct, \sigma_{d-1}, A_{d-1}, \ldots, \sigma_0, A_0, A_c)$.
\end{itemize}

\subsection{Verification}

$\calV$ takes as input a hash $m \in \Bn$, a public key $pk \in \Bn$ and a signature
\[(s, \sigma_d, oct, \sigma_{d-1}, A_{d-1}, \ldots, \sigma_0, A_0, A_c)\]
and verifies it as follows.

\begin{itemize}
\item Compute the hyper-tree index and random subset as $j, (x_1, \ldots, x_k) \gets \PORSsf(s, m)$.
\item Compute the $\PORST$ public key $p \gets \PORSTextract(x_1, \ldots, x_k, \sigma_d, oct)$.
\item If $p = \bot$, then abort and return $0$.
\item For $i \in \{d-1, \ldots, 0\}$ do the following:
    \begin{itemize}
    \item compute the $\WOTS$ public key $p \gets \WOTSextract(p, \sigma_i)$,
    \item set $j' \gets \lfloor j / 2^h \rfloor$,
    \item compute the Merkle root $p \gets \mextract_h(p, j - 2^h j', A_i)$,
    \item set $j \gets j'$.
    \end{itemize}
\item Compute the Merkle root $p \gets \mextract_c(p, j, A_c)$.
\item The result is $1$ if $p = pk$, and $0$ otherwise.
\end{itemize}

\subsection{Batch Signature}

$\calS_B$ takes as input a sequence of messages $(M_1, \ldots, M_i) \in \calM^i$ with $0 < i \le 2^b$ and a secret key $sk = (\seed, \salt)$ along with its secret cache, and outputs $i$ signatures $\sigma_j$, computed as follows.

\begin{itemize}
\item For $j \in \{1, \ldots, i\}$ compute the message digest $m_j \gets H^*(M_j)$.
\item For $j \in \{i+1, \ldots, 2^b\}$ set $m_j \gets m_1$.
\item Compute $m \gets \mroot_b(m_1, \ldots, m_{2^b})$.
\item Compute $\sigma \gets \calS(sk, m)$.
\item For $j \in \{1, \ldots, i\}$ the $j$-th signature is $\sigma_j \gets (j, \mauth_b(m_1, \ldots, m_{2^b}, j), \sigma)$.
\end{itemize}
%
For $b=0$, we simplify $\calS_B(sk, M)$ to $\calS(sk, H^*(M))$.

\subsection{Batch Verification}

$\calV_B$ takes as input a public key $pk$, a message $M \in \calM$ and a signature $(j, A, \sigma)$, and verifies it as follows.

\begin{itemize}
\item Compute the message digest $m \gets H^*(M)$.
\item Compute the Merkle root $m \gets \mextract_b(m, j, A)$.
\item The result is $\calV(pk, m, \sigma)$.
\end{itemize}
%
For $b=0$, we simplify $\calV_B(pk, M, \sigma)$ to $\calV(pk, H^*(M), \sigma)$.


\section{Proposed Instances}

We now propose concrete instances of parameters and primitives for \gravity{}.

\subsection{Parameters}

We propose the following parameters, common to all our proposed instances.
\begin{itemize}
\item Hash output $n = 256$ bits, to aim for 128 bits of security for collision-resistance, both classical and quantum.
\item Winternitz depth $w = 16$, a good trade-off between size and speed often chosen in similar constructions (XMSS, SPHINCS).
\item A PORS set size $t = 2^{16}$, here again a good trade-off between size and speed chosen in SPHINCS.
\end{itemize}
%
Given these, Table~\ref{tbl:gravity_params} gives the parameters of our three proposed instances:
\begin{itemize}
% instance "small"
\item \textbf{Instance S} produces signatures of at most $\num{12640}$ bytes and provides 128-bit security if no more than $2^{10}$ messages are signed.
This version is for use cases where a limited number of signatures is issued (firmware signatures, certificate authorities, and so on).

% instance "fast"
\item \textbf{Instance M} produces signatures of at most $\num{28929}$ bytes and provides 128-bit security if no more than $2^{50}$ messages are signed.
This version is suitable for most use cases.

% instance "big-fast"
\item \textbf{Instance L} produces signatures of at most $\num{35168}$ bytes and provides 128-bit security if no more than $2^{64}$ messages are signed.
This version is for use cases where a single secret key may issue more than $2^{50}$ signatures.

% keygen too slow
% instance "batching"
%\item \textbf{Instance B} was designed to sign up to $2^{40}$ \emph{batches} of messages (for example, up to $2^{64}$ messages if a batch includes $2^{24}$ messages). It produces signatures of at most $\num{20032}$ bytes.

\end{itemize}
If more than the authorized number of messages are signed, then the security level slowly degrades, and may eventually allow attackers to forge signatures efficiently.
Said forgeries would not be based on the real secret key, however, and could be distinguished from legitimately issued signatures.


\begin{table}
\begin{center}
\begin{tabular}{c|ccccc|cc|c}
\toprule
name      & $\log_2{t}$ & $k$ & $h$ & $d$ & $c$ & sig & sk & capacity \\
\midrule
S & 16          &  24 &   5 &   1 &  10 & \num{12640} & 64\,KiB & $2^{10}$ \\
M & 16          &  32 &   5 &   7 &  15 & \num{28929} & 2\,MiB & $2^{50}$ \\
L & 16          &  28 &   5 &  10 &  14 & \num{35168} & 1\,MiB & $2^{64}$ \\
\bottomrule
\end{tabular}
\end{center}
\caption{Proposed parameters for \gravity{}, suitable for 128 bits of post-quantum security.
The capacity is the number of messages (or batches of messages) that can be signed per key pair.
The value under ``sig'' is the maximal signature byte size.
The value under ``sk'' includes the 64-byte secret key plus the cached data.
Public keys are always 32-byte.}
\label{tbl:gravity_params}
\end{table}


\subsection{Primitives}
\label{sec:primitives}

\subsubsection{Hash Functions}

For the hash functions, we propose to use a 6-round version of Haraka-v2-256 as $F$, and a 6-round version of Haraka-v2-512 as $H$.
We extend the original Haraka-v2 construction with an additional round, whose constants are the following, computed with the same formula defined in~\cite{haraka}:
\begin{align*}
    RC_{40} = \texttt{2ff372380de7d31e367e4778848f2ad2} \\
    RC_{41} = \texttt{08d95c6acf74be8bee36b135b73bd58f} \\
    RC_{42} = \texttt{5880f434c9d6ee9866ae1838a3743e4a} \\
    RC_{43} = \texttt{593023f0aefabd99d0fdf4c79a9369bd} \\
    RC_{44} = \texttt{329ae3d1eb606e6fa5cc637b6f1ecb2a} \\
    RC_{45} = \texttt{e00207eb49e01594a4dc93d6cb7594ab} \\
    RC_{46} = \texttt{1caa0c4ff751c880942366a665208ef8} \\
    RC_{47} = \texttt{02f7f57fdb2dc1ddbd03239fe3e67e4a}
\end{align*}

For the general-purpose hash function $H^*$, we propose to use SHA-256, which is a NIST standard and widely available.


\subsubsection{Pseudorandom Function}

We propose a construction based on $\AES$ for $G$, valid as long as the parameters verify the constraints $c+dh \le 64$ and $\max(\ell,t) \le 2^{31}$.
More precisely, given a seed $s \in \Bn$ and an address $a = (i, j, \lambda) \in \calA$, we compute $G(s, a)$ as follows.
%
\begin{itemize}
\item Compute $P_0 \gets \encodeNbits{j}{64} || \encodeNbits{i}{32} || \encodeNbits{2\lambda}{32}$ and $P_1 \gets \encodeNbits{j}{64} || \encodeNbits{i}{32} || \encodeNbits{2\lambda+1}{32}$, where $\encodeNbits{x}{m}$ denotes the (bytewise) big-endian encoding of $x$ as an $m$-bit number.
\item The result is $\AES(s, P_0) || \AES(s, P_1)$, i.e.\ the encryption of $P_0$ and $P_1$ with key $s$.
\end{itemize}
%
We recall that due to the constraints on $(c, d, h)$, the in-layer index $j$ satisfies $0 \le j < 2^{c+dh} \le 2^{64}$, and the counter $\lambda$ satisfies $2\lambda+1 < 2^{32}$.

This construction is essentially $\AES$ in counter mode, except that we need two AES blocks for a 256-bit result.
%We use little-endian encoding of the counter $\lambda$ because this allows to use optimized addition arithmetic on processors supporting SSE2 instructions.
Note that the seed $s$ is the same throughout the hyper-tree, which allows a signer to cache the AES round keys.

